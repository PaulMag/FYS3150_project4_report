\documentclass[a4paper, 12pt, english]{article}

% Import packages:
\usepackage[utf8]{inputenc}
\usepackage[english]{babel} % change to english if english
\usepackage{graphicx, color}
\usepackage{parskip} % norwegian sections (skip line)
\usepackage{amsmath}
\usepackage{varioref} % fancy captions
\usepackage[margin=3cm]{geometry} % smaller margins
\usepackage{grffile} % Extended file name support for graphics, allows periods in filenames
\usepackage[small,bf]{caption}
\usepackage{hyperref}
\usepackage{fancyhdr}
\usepackage{algpseudocode}
\usepackage{parskip}
\usepackage{caption}

% For placeholder text, :D
\usepackage{lipsum}

% Packages for writing algorithms
\usepackage{algpseudocode}
\usepackage{algorithm}

\definecolor{darkgreen}{RGB}{0,135,0}

% Set code stuff:
\usepackage{listings}
\lstset{language=c++} % or python, java, etc...
\lstset{basicstyle=\ttfamily\scriptsize} % \small if short code
\lstset{frame=single} % creates the frame around code
\lstset{title=\lstname} % display name of file, not necessary
\lstset{keywordstyle=\color{red}\bfseries}
\lstset{commentstyle=\color{blue}}
\lstset{stringstyle =\color{darkgreen}}
\lstset{showspaces=false}
\lstset{showstringspaces=false}
\lstset{showtabs=false}
\lstset{breaklines=true}
\lstset{tabsize=4}

% Custom commands:
%\newcommand{\nameOfCommand}[numberOfArguments]{command}
\newcommand{\D}[1]{\ \mathrm{d}#1} % steps in integrals, ex: 4x \D{x} -> 4x dx
\newcommand{\E}[1]{\cdot 10^{#1}}  % exponents, ex: 1.4\E{3} -> 1.4*10^3
\newcommand{\U}[1]{\, \textrm{#1}} % display units prettily, ex: 15.4\U{m} -> 15.4 m
\newcommand{\degree}{\, ^\circ}    % make a degree symbol

\newcommand{\bilde}[3]{
	\begin{figure}[htbp]
		\centering
		\includegraphics[width=\textwidth]{#1}
		\caption{#3 \label{#2}}
	\end{figure}
}
\newcommand{\bildeto}[4]{
	\begin{figure}[htbp]
		\centering
		\includegraphics[width=0.96\textwidth]{#1}
		\includegraphics[width=0.96\textwidth]{#2}
		\caption{#4 \label{#3}}
	\end{figure}
}

% Commands for referencing
\newcommand{\refeq}[1]{(\textcolor{red}{\ref{eq:#1}})} % Red color when referencing equations.
\newcommand{\refig}[1]{\textcolor{blue}{\ref{fig:#1}}} % Blue color when referencing figures.
\newcommand{\reflst}[1]{(\textcolor{red}{\ref{lst:#1}})}
\newcommand{\reftab}[1]{\textcolor{blue}{\ref{tab:#1}}} % Blue color when referencing tables.

% Fancy ruler needed for titlepage
\newcommand{\HRule}{\rule{\linewidth}{0.5mm}}

% Command for most used vectors
\newcommand{\R}{\mathbf{r}}
\newcommand{\V}{\mathbf{v}}
\newcommand{\A}{\mathbf{a}}

% Opening:
\author{Kristoffer Brækken, Vedad Hodzic, Paul Magnus Sørensen-Clark}

% Begin document:
\begin{document}

\begin{titlepage}
    \thispagestyle{empty}
    % Explanation of \\[]-syntax:
% In the square brackets you can define a certain vertical space
% which is inserted in the text. Convenient for fancy formatting.

\begin{center}
    \LARGE University of Oslo\\[1.5cm]
    \Large FYS3150 - Project 4 \\ Computational physics\\[0.5cm]

    \HRule \\[0.4cm]

    % Title of project
    { \huge \bfseries Diffusion of neurotransmitters in the synaptic cleft \\[0.4cm] }

    \HRule \\[1.5cm]

    % Spaces between lines are there for line breaks
    \large Kristoffer Brækken, \emph{jaremikb}

    \large Paul Magnus Sørensen-Clark, \emph{paulms}

    \large Vedad Hodzic, \emph{vedadh}

    \vfill

    {\large \today}
\end{center}

\end{titlepage}

\begin{abstract}
    \lipsum[1]
\end{abstract}

\section{Introduction}
The purpose of this exercise is to simulate the diffusion of signal
molecules between neurons, both analytically and numerically. These
molecules are transmitted from a presynaptic membrane and captured
by a postsynaptic membrane. The space in between these membranes is
the synaptic cleft, and this is the area we will simulate.

This diffusion of molecules can be described via the diffusion
equation \[ \frac{\partial u}{\partial t} = D\frac{\partial^2
u}{\partial x^2}. \] In order to describe the physical situation
a numerical integration scheme can be implemented.

Solving this particular equation can provide challenges considering
the double spatial derivative on the right hand side. This can be
avoided via clever use of tridiagonal matrices.

The problem is solved via the explicit Forward Euler scheme, the
implicit Backward Euler scheme as well as the implicit Crank
Nicolson scheme. At later stages, an error analysis comparing the
three schemes is included to highlight potential error and to
illustrate areas of stability.

Each integration scheme produces an error and has an area of
stability. When comparing different numerical methods, these two
are key aspects.

A closed form solution to the problem is also found in order to
compare the numerical results and illustrate errors.


\section{Theory}

\section{Algorithm}
% \subsection{Forward Euler}

With the forward scheme we have
\begin{align*}
    u_t &= u_{xx} \\
    \frac{u(x_i, t_{j+1}) - u(x_i, t_j)}{\Delta t}
    &= \frac{u(x_{i+1}, t_j) - 2u(x_i, t_j) + u(x_{i-1}, t_j)}{\Delta x^2}.
\end{align*}
This we can simply solve for $u(x_i, t_{j+1}$ to get
\begin{align*}
    u(x_i, t_{j+1}) \\
    &= u(x_i, t_j)
    +  \left( u(x_{i+1}, t_j) - 2u(x_i, t_j) + u(x_{i-1}, t_j) \right)
       \frac{\Delta t}{\Delta x^2}.
\end{align*}
Now we can find any point on $u_{j+1}$ independently of each other
as long as we know the previous timestep, $u_j$. We can do a simple
loop through $u$ to set all the new positions. The endpoints need
to be set outside the loop. In this case the endpoints are fixed,
so we need to simply not change them. To solve for several time
steps this loop can be placed in an outer time loop.
\begin{algorithm}
    \For{j = 1, 2, 3, \cdots}
        \For{i = 2, 3, \cdots , n-2, n-1}
            \State $ u_{new}(x_i, t_{j+1})
                = u(x_i, t_j)
                + \left( u(x_{i+1}, t_j) - 2u(x_i, t_j) + u(x_{i-1}, t_j) \right
                    \frac{\Delta t}{\Delta x^2} $
        \EndFor
        \State $ u = u_{new} $
    \EndFor
\end{algorithm}


\subsection{Backward Euler}


\subsection{Crank-Nicelson}


\section{Results}

In order to determine the numerical accuracy of our results, we need to be able
to compare it with a closed form solution. We're interested in finding a
solution for the one-dimensional diffusion equation
%
\begin{equation}
	\frac{\partial u(x,t)}{\partial t} = \frac{\partial^2 u(x,t)}{\partial x^2},
	\hspace{1cm} x \in [0,d], \ t > 0,
	\label{eq:diffusioneq}
\end{equation}
%
where $d = 1$, boundary conditions 
%
\begin{align}
	u(0,t) = 1, \hspace{1cm} u(1,t) = 0
	\label{eq:bc}
\end{align}
%
and initial condition 
%
\begin{equation*}
	u(x,0) = 0.
\end{equation*}
%
It is easier to solve such an equation, both numerically and on a closed form,
by the function being zero at both ends. We can achieve this by first finding the
stationary solution, $u_s(x)$. This means that there is no time development of the
function, so \refeq{diffusioneq} transforms to
%
\begin{equation}
	u_s(x) = \frac{\partial^2 u(x,t)}{\partial x^2} = 0.
	\label{eq:stationary}
\end{equation}
%
It is easy to see that for \refeq{stationary} to be fulfilled, our solution must
be on the form
%
\begin{equation*}
	u(x,t) = Ax + B.
\end{equation*}
%
Applying the boundary conditions \refeq{bc} we get
%
\begin{align*}
	u_s(0) &= B = 1 \\
	u_s(1) &= A + 1 = 0 \ \Rightarrow \ A = -1
\end{align*}
%
which gives us
%
\begin{equation*}
	u_s(x) = 1 - x.
%	\label{eq:us}
\end{equation*}
%
We see now that we can define a new function as $v(x) = u(x) - u_s(x)$, which
has boundary conditions and initial condition
%
\begin{equation}
	v(0) = u(0) - u_s(0) = 1 - 1 = 0
	\label{eq:bc0new}
\end{equation}
\begin{equation}
	v(1) = u(1) - u_s(1) = 0 - 0 = 0
	\label{eq:bc1new}
\end{equation}
\begin{align}
	v(x,0) &= u(x,0) - u_s(x) \nonumber \\
	v(x,0) &= x - 1. \label{eq:newInitial}
\end{align}
%
This means that solving the equation
%
\begin{equation*}
	\frac{\partial v(x,t)}{\partial t} = \frac{\partial^2 v(x,t)}{\partial x^2}
\end{equation*}
%
with boundary conditions
%
\begin{equation*}
	v(0,t) = 0, \hspace{1cm} v(1,t) = 0
\end{equation*}
%
gives the same solution as solving \refeq{diffusioneq} with boundary conditions
\refeq{bc}.

We assume the solution $v(x,t)$ is separable, meaning we can write it as
$v(x,t) = F(x)G(t)$.
This gives us
%
\begin{align}
	\frac{\partial F(x)G(t)}{\partial t} &= \frac{\partial^2 F(x)G(t)}{\partial
	x^2} \nonumber \\
	F(x)G'(t) &= \frac{\partial F'(x)G(t)}{\partial x} \nonumber \\
	F(x)G'(t) &= F''(x)G(t) \nonumber \\
	\frac{F''(x)}{F(x)} &= \frac{G'(t)}{G(t)}.
	\label{eq:separate}
\end{align}
%
Since \refeq{separate} needs to be fulfilled for all $x,t$ we require the two
sides being equal to a constant, $-\lambda^2$.
%
\begin{align}
	\frac{F''(x)}{F(x)} &= -\lambda^2 \nonumber \\
	F''(x) + \lambda^2F(x) &= 0
	\label{eq:sepF}
\end{align}
%
\begin{align}
	\frac{G'(t)}{G(t)} &= -\lambda^2 \nonumber \\
	G'(t) &= -\lambda^2G(t).
	\label{eq:sepG}
\end{align}
%
The first equation, \refeq{sepF} is a second-order homogeneous linear
differential equation with constant coefficients. We know the solution is on the
form
%
\begin{equation}
	F(x) = c_1e^{r_1x} + c_2e^{r_2x},
	\label{eq:F}
\end{equation}
%
with $r_1$, $r_2$ being the roots of the auxiliary equation
%
\begin{align}
	r^2 + \lambda^2 &= 0 \nonumber \\
	r &= \pm i\lambda. \label{eq:roots}
\end{align}
%
Inserting \refeq{roots} in \refeq{F} gives
%
\begin{align*}
	F(x) &= c_1e^{i\lambda x} + c_2e^{-i\lambda x} \\
	F(x) &= c_1\left[ \cos{(\lambda x)} + i\sin{(\lambda x)} \right] + c_2\left[
	\cos{(\lambda x)} - i\sin{(\lambda x)} \right] \\
	F(x) &= (c_1 + c_2)\cos{(\lambda x)} + i(c_1 - c_2)\sin{(\lambda x)} \\
	F(x) &= A\sin{(\lambda x)} + B\cos{(\lambda x)}.
\end{align*}
%
The second equation, \refeq{sepG}, is a first order homogeneous ODE with
constant coefficients. We solve it by integrating
%
\begin{align*}
	\frac{\mathrm{d}G(t)}{\mathrm{d}t} &= -\lambda^2 G(t) \\
	\frac{\mathrm{d}G(t)}{G(t)} &= - \lambda^2 dt \\
	\int \frac{1}{G(t)} \ \mathrm{d}G(t) &= -\lambda^2 \int \mathrm{d}t \\
	\ln{G(t)} &= -\lambda^2 t + d_1 \\
	G(t) &= e^{-\lambda^2 t + d_1} = e^{d_1}e^{-\lambda^2 t} \\
	G(t) &= Ce^{-\lambda^2 t}.
\end{align*}
%
We can determine the coefficient $A,B,C$ from the boundary conditions
\refeq{bc0new}, \refeq{bc1new}.
%
\begin{align*}
	v(0,t) = F(0)G(t) &= 0 \\
	BCe^{-\lambda^2 t} &= 0 \\
	B = 0 \ \vee \ C &= 0.
%	B = 0 \hspace{1cm} \mathrm{or} \hspace{1cm} C &= 0.
\end{align*}
%
The case where $C = 0$ is a trivial case where there is no time development, so 
we are not interested in this solution. We let $B = 0$, which gives us
%
\begin{align*}
	v(1,t) &= A\sin{(\lambda)} \ Ce^{-\lambda^2 t} = 0 \\
	&\Rightarrow \lambda = n\pi.
\end{align*}
%
One solution of $v(x,t)$ is then
%
\begin{equation*}
	v(x,t) =  D_n\sin{(n\pi x)}e^{-n^2\pi^2 t},
\end{equation*}
%
where we have used $A_nC_n = D_n$.
However, there are infinitely many possible values for $n$, which gives us an
infinite number of solutions. Because the diffusion equation is linear, a
superposition of solutions is also a solution for $v(x,t)$, so
%
\begin{equation*}
	\sum\limits_{n=1}^{\infty} D_n\sin{(n\pi x)}e^{-n^2\pi^2 t}.
\end{equation*}
%
We determine the coefficients $D_n$ from the initial condition
\refeq{newInitial}
%
\begin{equation*}
	v(x,0) = \sum\limits_{n=1}^{\infty} D_n\sin{(n\pi x)} = x - 1.
\end{equation*}
%
We recognize that the coefficients $D_n$ are the Fourier coefficients for
$x - 1$. We can then determine $D_n$ from the Fourier series
%
\begin{align*}
	D_n &= 2\int\limits_0^1 (x - 1)\sin{(n\pi x)} \ \mathrm{d}x \\
	&= 2 \int\limits_0^1 \left[ x\sin{(n\pi x)} - \sin{(n\pi x)} \right] \
	\mathrm{d}x \\
	&= 2 \left( \left[ -x\cos{(n\pi x)}\frac{1}{n\pi} \right]_0^1 
	+ \int\limits_0^1\frac{1}{n\pi}\cos{(n\pi x)} \ \mathrm{d}x - \int\limits_0^1
	\sin{(n\pi x)} \ \mathrm{d}x \right) \\
	&= 2 \left( -\frac{1}{n\pi}\cos{(n\pi)} + \frac{1}{n^2\pi^2}\left[
		\sin{(n\pi x)}
	\right]_0^1 + \left[ \frac{1}{n\pi}\cos{(n\pi x)} \right]_0^1
	\right) \\
	&= 2 \left( \frac{\sin{(n\pi)}}{n^2\pi^2} - \frac{1}{n\pi} \right) \\
	&= -\frac{2}{n\pi}.
\end{align*}
%
Now that we have found the coefficients $D_n$, we have our solution $v(x,t)$,
and thus we can find $u(x,t)$.
%
\begin{align*}
	v(x,t) &= u(x,t) - u_s(x) \\
	u(x,t) &= v(x,t) + u_s(x) \\
	u(x,t) &= 1 - x - \sum\limits_{n=1}^{\infty}
	\frac{2}{n\pi}\sin{(n\pi x)}e^{-n^2\pi^2 t}.
\end{align*}
%


\section{Discussion}

\section{Source code}

\end{document}
