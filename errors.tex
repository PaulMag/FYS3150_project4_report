Now we want to look at the size of the actual error of the three schemes. We 
look at absolute differences between the numerical methods and the analytical 
solution.

\bildeto{figures/error_100_0.05_0.1.png}{figures/error_100_1_0.1.png}
{fig:error_stable}{$\alpha=0.1$ Forward Euler is well within its stability 
condition. Top: Before stationairy state. Bottom: After stationairy state.}

\bildeto{figures/error_100_0.05_0.5.png}{figures/error_100_1_0.5.png}
{fig:error_limit}{$\alpha=0.5$ Forward Euler is on the edge of being stable. 
Top: Before stationairy state. Bottom: After stationairy state.}

The first error analysis is from $\alpha = 0.1$, so that all the methods are 
well within the stable range. From figure \refig{error_stable} we can see that 
the forward scheme ends up with the smalles error, the implicit backward scheme 
gives the largest error, and Cran-Nicolson simply appear to be a rough average 
of the other two.

The next analysis is from $\alpha = 0.5$ such that the explicit method is just 
within its stable range. In figure \refig{error_limit} is the result. Both 
Forward and Backward Euler has increased their errors by an order of magnitude, 
but Forward Euler increased the most, relative to the previous case. It is 
still stable, but it behaves weirdly, suggesting it is on the verge of failing. 
The Crank-Nicolson method appears to have not been affected at all by the 
increase in timestep, suggesting this is a very stable method.

When increasing $\alpha$ from $0.5$ to $0.51$ the Forward Euler error 
immediatelly blew out of proportions and was thus useless, while the implicit 
methods only changed slightly.