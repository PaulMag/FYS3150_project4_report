
In order to determine the numerical accuracy of our results, we need to be able
to compare it with a closed form solution. We're interested in finding a
solution for the one-dimensional diffusion equation
%
\begin{equation}
	\frac{\partial u(x,t)}{\partial t} = \frac{\partial^2 u(x,t)}{\partial x^2},
	\hspace{1cm} x \in [0,d], \ t > 0,
	\label{eq:diffusioneq}
\end{equation}
%
where $d = 1$, boundary conditions 
%
\begin{align}
	u(0,t) = 1, \hspace{1cm} u(1,t) = 0
	\label{eq:bc}
\end{align}
%
and initial condition 
%
\begin{equation*}
	u(x,0) = 0.
\end{equation*}
%
It is easier to solve such an equation, both numerically and on a closed form,
by the function being zero at both ends. We can achieve this by first finding the
stationary solution, $u_s(x)$. This means that there is no time development of the
function, so \refeq{diffusioneq} transforms to
%
\begin{equation}
	u_s(x) = \frac{\partial^2 u(x,t)}{\partial x^2} = 0.
	\label{eq:stationary}
\end{equation}
%
It is easy to see that for \refeq{stationary} to be fulfilled, our solution must
be on the form
%
\begin{equation*}
	u(x,t) = Ax + B.
\end{equation*}
%
Applying the boundary conditions \refeq{bc} we get
%
\begin{align*}
	u_s(0) &= B = 1 \\
	u_s(1) &= A + 1 = 0 \ \Rightarrow \ A = -1
\end{align*}
%
which gives us
%
\begin{equation*}
	u_s(x) = 1 - x.
%	\label{eq:us}
\end{equation*}
%
We see now that we can define a new function as $v(x) = u(x) - u_s(x)$, which
has boundary conditions and initial condition
%
\begin{equation}
	v(0) = u(0) - u_s(0) = 1 - 1 = 0
	\label{eq:bc0new}
\end{equation}
\begin{equation}
	v(1) = u(1) - u_s(1) = 0 - 0 = 0
	\label{eq:bc1new}
\end{equation}
\begin{align}
	v(x,0) &= u(x,0) - u_s(x) \nonumber \\
	v(x,0) &= x - 1. \label{eq:newInitial}
\end{align}
%
This means that solving the equation
%
\begin{equation*}
	\frac{\partial v(x,t)}{\partial t} = \frac{\partial^2 v(x,t)}{\partial x^2}
\end{equation*}
%
with boundary conditions
%
\begin{equation*}
	v(0,t) = 0, \hspace{1cm} v(1,t) = 0
\end{equation*}
%
gives the same solution as solving \refeq{diffusioneq} with boundary conditions
\refeq{bc}.

We assume the solution $v(x,t)$ is separable, meaning we can write it as
$v(x,t) = F(x)G(t)$.
This gives us
%
\begin{align}
	\frac{\partial F(x)G(t)}{\partial t} &= \frac{\partial^2 F(x)G(t)}{\partial
	x^2} \nonumber \\
	F(x)G'(t) &= \frac{\partial F'(x)G(t)}{\partial x} \nonumber \\
	F(x)G'(t) &= F''(x)G(t) \nonumber \\
	\frac{F''(x)}{F(x)} &= \frac{G'(t)}{G(t)}.
	\label{eq:separate}
\end{align}
%
Since \refeq{separate} needs to be fulfilled for all $x,t$ we require the two
sides being equal to a constant, $-\lambda^2$.
%
\begin{align}
	\frac{F''(x)}{F(x)} &= -\lambda^2 \nonumber \\
	F''(x) + \lambda^2F(x) &= 0
	\label{eq:sepF}
\end{align}
%
\begin{align}
	\frac{G'(t)}{G(t)} &= -\lambda^2 \nonumber \\
	G'(t) &= -\lambda^2G(t).
	\label{eq:sepG}
\end{align}
%
The first equation, \refeq{sepF} is a second-order homogeneous linear
differential equation with constant coefficients. We know the solution is on the
form
%
\begin{equation}
	F(x) = c_1e^{r_1x} + c_2e^{r_2x},
	\label{eq:F}
\end{equation}
%
with $r_1$, $r_2$ being the roots of the auxiliary equation
%
\begin{align}
	r^2 + \lambda^2 &= 0 \nonumber \\
	r &= \pm i\lambda. \label{eq:roots}
\end{align}
%
Inserting \refeq{roots} in \refeq{F} gives
%
\begin{align*}
	F(x) &= c_1e^{i\lambda x} + c_2e^{-i\lambda x} \\
	F(x) &= c_1\left[ \cos{(\lambda x)} + i\sin{(\lambda x)} \right] + c_2\left[
	\cos{(\lambda x)} - i\sin{(\lambda x)} \right] \\
	F(x) &= (c_1 + c_2)\cos{(\lambda x)} + i(c_1 - c_2)\sin{(\lambda x)} \\
	F(x) &= A\sin{(\lambda x)} + B\cos{(\lambda x)}.
\end{align*}
%
The second equation, \refeq{sepG}, is a first order homogeneous ODE with
constant coefficients. We solve it by integrating
%
\begin{align*}
	\frac{\mathrm{d}G(t)}{\mathrm{d}t} &= -\lambda^2 G(t) \\
	\frac{\mathrm{d}G(t)}{G(t)} &= - \lambda^2 dt \\
	\int \frac{1}{G(t)} \ \mathrm{d}G(t) &= -\lambda^2 \int \mathrm{d}t \\
	\ln{G(t)} &= -\lambda^2 t + d_1 \\
	G(t) &= e^{-\lambda^2 t + d_1} = e^{d_1}e^{-\lambda^2 t} \\
	G(t) &= Ce^{-\lambda^2 t}.
\end{align*}
%
We can determine the coefficient $A,B,C$ from the boundary conditions
\refeq{bc0new}, \refeq{bc1new}.
%
\begin{align*}
	v(0,t) = F(0)G(t) &= 0 \\
	BCe^{-\lambda^2 t} &= 0 \\
	B = 0 \ \vee \ C &= 0.
%	B = 0 \hspace{1cm} \mathrm{or} \hspace{1cm} C &= 0.
\end{align*}
%
The case where $C = 0$ is a trivial case where there is no time development, so 
we are not interested in this solution. We let $B = 0$, which gives us
%
\begin{align*}
	v(1,t) &= A\sin{(\lambda)} \ Ce^{-\lambda^2 t} = 0 \\
	&\Rightarrow \lambda = n\pi.
\end{align*}
%
One solution of $v(x,t)$ is then
%
\begin{equation*}
	v(x,t) =  D_n\sin{(n\pi x)}e^{-n^2\pi^2 t},
\end{equation*}
%
where we have used $A_nC_n = D_n$.
However, there are infinitely many possible values for $n$, which gives us an
infinite number of solutions. Because the diffusion equation is linear, a
superposition of solutions is also a solution for $v(x,t)$, so
%
\begin{equation*}
	\sum\limits_{n=1}^{\infty} D_n\sin{(n\pi x)}e^{-n^2\pi^2 t}.
\end{equation*}
%
We determine the coefficients $D_n$ from the initial condition
\refeq{newInitial}
%
\begin{equation*}
	v(x,0) = \sum\limits_{n=1}^{\infty} D_n\sin{(n\pi x)} = x - 1.
\end{equation*}
%
We recognize that the coefficients $D_n$ are the Fourier coefficients for
$x - 1$. We can then determine $D_n$ from the Fourier series
%
\begin{align*}
	D_n &= 2\int\limits_0^1 (x - 1)\sin{(n\pi x)} \ \mathrm{d}x \\
	&= 2 \int\limits_0^1 \left[ x\sin{(n\pi x)} - \sin{(n\pi x)} \right] \
	\mathrm{d}x \\
	&= 2 \left( \left[ -x\cos{(n\pi x)}\frac{1}{n\pi} \right]_0^1 
	+ \int\limits_0^1\frac{1}{n\pi}\cos{(n\pi x)} \ \mathrm{d}x - \int\limits_0^1
	\sin{(n\pi x)} \ \mathrm{d}x \right) \\
	&= 2 \left( -\frac{1}{n\pi}\cos{(n\pi)} + \frac{1}{n^2\pi^2}\left[
		\sin{(n\pi x)}
	\right]_0^1 + \left[ \frac{1}{n\pi}\cos{(n\pi x)} \right]_0^1
	\right) \\
	&= 2 \left( \frac{\sin{(n\pi)}}{n^2\pi^2} - \frac{1}{n\pi} \right) \\
	&= -\frac{2}{n\pi}.
\end{align*}
%
Now that we have found the coefficients $D_n$, we have our solution $v(x,t)$,
and thus we can find $u(x,t)$.
%
\begin{align*}
	v(x,t) &= u(x,t) - u_s(x) \\
	u(x,t) &= v(x,t) + u_s(x) \\
	u(x,t) &= 1 - x - \sum\limits_{n=1}^{\infty}
	\frac{2}{n\pi}\sin{(n\pi x)}e^{-n^2\pi^2 t}.
\end{align*}
%
