The purpose of this exercise is to simulate the diffusion of signal
molecules between neurons, both analytically and numerically. These
molecules are transmitted from a presynaptic membrane and captured
by a postsynaptic membrane. The space in between these membranes is
the synaptic cleft, and this is the area we will simulate.

To simplify the problem we will approximate this event by assuming
that all movement happens in the same direction, so that we will
only have one spatial dimension plus time. I.e. we are thinking of
the membranes as being infinitely large and parallel, so that on
average, there is no total movement parallel to the membranes, only
parallel to a normal line between them. This approximation is valid
if both membranes are very large and flat compared to the width of
the synaptic cleft.

The spatial direction is called $x$ and time $t$. The position of
the presynaptic membrane is $x=0$ and the postsynaptic membrane is
at $x=1$. The synaptic cleft is the area between, $x \in [0,1]$. We
will not look at discrete signal molecules, but at a floating
density of molecules at a given point. This is a valid
approximation if there are very many molecules. The density at a
given position is called $u(x)$. The presynaptic membrane will
transmit signal molecules in such a way that $u(0)=11$ always. The
postsynaptic membrane absorbs molecules in such a way that $u(1)=0$
always.

At $t=0$ the synaptic cleft is empty, so $u(x)=0$  for $0 < x \leq
1$. When $t > 0$ the cleft will begin to fill up, until it
eventually reaches a steady state where the rate at which molecules
are removed from the system is the same as the rate at which they
are added.
