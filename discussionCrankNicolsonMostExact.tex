Observe also that in figure \refig{zoomed_schemes}, the scheme
closest to the closed form solution is the Crank Nicolson scheme.
Via utilizing the mean of the two algorithms, Forward Euler and
Backward Euler, the Crank Nicolson scheme has ``inherited'' the
best properties of the other two methods. This gives it stability
from the implicit Euler method and accuracy from the explicit Euler
method.

In test case 1, the difference in accuracy between the explicit
Euler and the implicit methods is much more distinguishable than
for test case 2. This suggests a greater dependence on the number
of time steps for the FE method than for the others.

One must also mention that implicit schemes only become more
practical when dealing with stiff equations. That is, problems
which have rapid changes. Still, implementing the Crank Nicolson
scheme means that for problems as the one describes here, very
small spatial scales can be simulated, without the need of a tiny
time step. This can ease computation time vastly.

All things considered, the Crank Nicolson scheme provides the best
results for this particular problem.
