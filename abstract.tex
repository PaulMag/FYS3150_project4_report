
Communication in the brain relies heavily on the diffusion of neurotransmitters
across the synaptic cleft, that separates the cell membranes of two cells.
A solution to the diffusion equation can be used to describe this process, so 
we apply three numerical methods for solving the one-dimensional case. A closed
form solution is used to analyze the numerical accuracy of the explicit
forward Euler, and the implicit backward Euler and Crank-Nicolson scheme. It
shows that the explicit forward Euler algorithm yields the most accurate
results at the loss of stability, while the backward case is stable, but with
poor accuracy. The Crank-Nicolson scheme applies both
the explicit forward Euler and the implicit backward Euler, resulting in a
stable method, with an accuracy in between the two Euler schemes.
The implicit Crank-Nicolson scheme is
then preferred in applications where a very small spatial step length is required,
as to avoid compensating with an even smaller temporal step length --- which will
induce a penalty in computation time.

% This was part of the unfinished first draft
%The dominant means of communication between neurons in the brain depend on
%diffusion of signal molecules called \emph{neurotransmitters} across the
%synaptic cleft that separates the cell membranes of two cells. The
%neurotransmitters are released by vesicles in the axon terminal, upon which
%receptors on the postsynaptic side receive the signal. Further insight in this
%process can be helpful within the field of medicine for treating patients that
%suffer from various illnesses caused by impaired communication in the brain.
%
%We will try to simulate the process of diffusion of neurotransmitters across the
%synaptic cleft by solving the diffusion equation. We do the simulation by
%solving the one-dimensional case using three numerical methods, then 
%
%We will however, in our work, mostly consider the differences in numerical
%accuracy between various numerical methods that solve the one-dimensional case.

