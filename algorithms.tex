\subsection{Forward Euler}

With the forward scheme we have
\begin{align*}
    u_t &= u_{xx} \\
    \frac{u(x_i, t_{j+1}) - u(x_i, t_j)}{\Delta t}
    &= \frac{u(x_{i+1}, t_j) - 2u(x_i, t_j) + u(x_{i-1}, t_j)}{\Delta x^2}.
\end{align*}
This we can simply solve for $u(x_i, t_{j+1})$ to get
\begin{align*}
    u(x_i, t_{j+1})
    &= u(x_i, t_j)
    +  \left[ u(x_{i+1}, t_j) - 2u(x_i, t_j) + u(x_{i-1}, t_j) \right]
       \frac{\Delta t}{\Delta x^2}.
\end{align*}
Now we can find any point on $u_{j+1}$ independently of each other
as long as we know the previous timestep, $u_j$. We can do a simple
loop through $u$ to set all the new positions. The endpoints need
to be set outside the loop. In this case the endpoints are fixed,
so we need to simply not change them. To solve for several time
steps this loop can be placed in an outer time loop.
\begin{algorithmic}
    \State $ \alpha = \frac{\Delta t}{\Delta x^2} $
    \For{j = 1, 2, 3, \dots}
        \For{i = 2, 3, \dots , n-2, n-1}
            \State $ u_{new}(x_i)
            = u(x_i)
            + \left[
              u(x_{i+1}) - 2u(x_i) + u(x_{i-1})
              \right] \alpha $
        \EndFor
        \State $ u = u_{new} $
    \EndFor
\end{algorithmic}



\subsection{Backward Euler}

With the backwards scheme we have
\begin{align*}
    u_t &= u_{xx} \\
    \frac{u(x_i, t_j) - u(x_i, t_{j-1})}{\Delta t}
    &= \frac{u(x_{i+1}, t_j) - 2u(x_i, t_j) + u(x_{i-1}, t_j)}{\Delta x^2}.
\end{align*}
Now we are working with the timesteps $j-1$ and $j$, so we might as well add 
$1$ so that we can keep calling the next step $j+1$. If we also implement 
$\alpha$ we get
\begin{align*}
    u(x_i, t_{j+1}) - u(x_i, t_j)
    &= \left[
       u(x_{i+1}, t_{j+1}) - 2u(x_i, t_{j+1}) + u(x_{i-1}, t_{j+1}) 
       \right] \alpha \\       
    - u(x_i, t_j)
    &= - u(x_i, t_{j+1})
       + \left[ u(x_{i+1}, t_{j+1}) - 2u(x_i, t_{j+1}) + u(x_{i-1}, t_{j+1}) 
        \right] \alpha \\   
    u(x_i, t_j)
    &=         - \alpha \cdot u(x_{i-1}, t_{j+1})
       + (1 + 2 \alpha) \cdot u(x_{i+1}, t_{j+1})   
               - \alpha \cdot u(x_{i+1}, t_{j+1})
\end{align*}
Now we cannot find the next position $u_{j+1}$ in a straightforward manner, 
since there are several dependencies on $u_{j+1}$, but we can set this equation 
up as a matrix equation and solve for $u_{j+1}$ through row reduction.
$$
\begin{bmatrix}
    1+2\alpha &   -\alpha &         0 &    \cdots &         0 \\
      -\alpha & 1+2\alpha &   -\alpha &    \ddots &    \vdots \\
            0 &   -\alpha & 1+2\alpha &    \ddots &         0 \\
       \vdots &    \ddots &    \ddots &    \ddots &   -\alpha \\
            0 &    \cdots &         0 &   -\alpha & 1+2\alpha
\end{bmatrix}
\cdot
\begin{bmatrix}
    u_{1,j+1} \\ u_{2,j+1} \\ u_{3,j+1} \\ \vdots \\ u_{n,j+1}
\end{bmatrix}
=
\begin{bmatrix}
    u_{1,j} \\ u_{2,j} \\ u_{3,j} \\ \vdots \\ u_{n,j}
\end{bmatrix}
$$
The matrix being tridiagonal saves us some work. We only need to perform $2n$ 
iterations to reduce it to a diagonal matrix so we can solve the equation. Let 
us define $a$, $b$ and $c$ as being the values that fill the lower, middle, and 
upper diagonal, respectively. First we will do a forward substitution to set 
the lower diagonal (the $a$'s) to $0$. Then the elements on the middle diagonal 
and $u_j$ will change. Let us put a hat on the changed elements. Note that 
$\hat{b}$ must be a vector now. The first iteration must happen outside the loop
to maintain the correct boundary condition, i.e. not changing the boundaries. 
This time we define the first element to be $u(x_0)$.
\begin{algorithmic}
    \State $ \hat{b}_1 = b $
    \State $ \hat{u}(x_1) = u(x_1) - u(x_0) \cdot a $
    \For{i = 2, 3, \dots, n-1, n}
        \State $ \hat{b}_i
                 = b    - c \cdot a / \hat{b}_{i-1} $
        \State $ \hat{u}(x_i)
                 = u(x_i) - \hat{u}(x_{i-1}) \cdot a / \hat{b}_{i-1} $
    \EndFor
\end{algorithmic}
Next comes a bacward substitution where we set the upper diagonal to be $0$ and 
then solve for the nest timestep $u_\textrm{new}$. Again, the first iteration
must happen outside the loop because it has different conditions than the 
others.
\begin{algorithmic}
    \State $ u_\textrm{new}(x_n) = u(x_n) / \hat{b}_n $
    \For{i = n-1, n-2, \dots, 2, 1}
        \State $ u_\textrm{new}(x_i)
                 = \left[ \hat{u}(x_i) - u_\textrm{new}(x_n) \right] / \hat{b} $
    \EndFor
\end{algorithmic}
This entire process can then be placed inside a time loop and repeated as many 
times as we like.



\subsection{Crank-Nicolson}
