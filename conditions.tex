The spatial direction is called $x$ and time $t$. The position of
the presynaptic membrane is $x=0$ and the postsynaptic membrane is
at $x=1$. The synaptic cleft is the area between, $x \in [0,1]$. We
will not look at discrete signal molecules, but at a floating
density of molecules at a given point. This is a valid
approximation if there are very many molecules. The density at a
given position is called $u(x)$. The presynaptic membrane will
transmit signal molecules in such a way that $u(0)=1$ always. The
postsynaptic membrane absorbs molecules in such a way that $u(1)=0$
always.

At $t=0$ the synaptic cleft is empty, so $u(x)=0$  for $0 < x \leq
1$. When $t > 0$ the cleft will begin to fill up, until it
eventually reaches a steady state $u_s(x)$ where the rate at which molecules
are removed from the system is the same as the rate at which they
are added.

The nature of the movement of the signal molecules, or the flow of the 
moleculels density, is by diffusion, so we will need to solve the diffusion 
equation with the inital and boundary conditions we have discussed so far.